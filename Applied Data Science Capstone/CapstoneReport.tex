\documentclass{article}
\usepackage{hyperref}
\begin{document}

\title{Car Accident Severity Prediction}
\author{Sudipto Ghosh}
\date{September 13, 2020}
\maketitle

\section{Introduction}

\subsection{Background}
The Open Data Program makes the data generated by the City of Seattle has been openly available to the public for the purpose of increasing the quality of life for the residents, increasing transparency, accountability and comparability, promoting economic development and research, and improving internal performance management.

The Traffic Records Group, Traffic Management Division, Seattle Department of Transportation, provides data for all collisions and crashes that have occurred in the state from 2004 to the present day. The data is updated weekly and can be found at the Seattle Open GeoData Portal\footnote[1]{\href{https://data-seattlecitygis.opendata.arcgis.com/}{https://data-seattlecitygis.opendata.arcgis.com/}}.

\subsection{Motivation}
We can exploit this data to extract vital features that would enable us to end up with a good model that would enable the prediction of the severity of future accidents that take place in the state. This would further enable the Department of Transportation to prioritise their SOPs and channel their energy to ensure that fewer fatalities result in automobile collisions.

\section{Data}

\subsection{Data Understanding}
The dataset is available as comma-separated values (CSV) files, KML files, and ESRI shapefiles that can be downloaded from the Seattle Open GeoData Portal. The data is also available from RESTful API services in formats such as GeoJSON. We download the dataset to our project directory and take a look at the data types and the dimensionality of the data. We could see that the dataset contains 221,389 records and 40 fields.

The metadata of the dataset can be found from the website of the Seattle Department of Transportation\footnote[2]{\href{https://www.seattle.gov/Documents/Departments/SDOT/GIS/Collisions\_OD.pdf}{https://www.seattle.gov/Documents/Departments/SDOT/GIS/Collisions\_OD.pdf}}.

On reading the dataset summary, we can determine the description of each of the fields and their possible values. The data contains several categorical fields and corresponding descriptions which could help us in further analysis. We made an attempt at understanding the data in terms of the fields that we shall take into account for later stages of model building.

\begin{table}
  \centering
  \begin{tabular}{|c | c|} 
  \hline
  Field Name & Description\\
  \hline
  X & Longitude (in degree decimal)\\
  Y & Latitude (in degree decimal)\\
  OBJECTID & ESRI unique identifier\\
  INCKEY & Unique key for the incident\\
  COLDETKEY & Secondary key for the incident\\
  REPORTNO & NA\\
  STATUS & NA\\
  ADDRTYPE & Collision address type: [Alley, Block, Intersection]\\
  INTKEY & Key to the intersection associated with a collision \\
  LOCATION & Description of the general location of the collision \\
  EXCEPTRSNCODE & NA\\
  EXCEPTRSNDESC & NA\\
  SEVERITYCODE & Code corresponding to the severity of the collision\\
  SEVERITYDESC & Detailed description of the severity of the collision\\
  COLLISIONTYPE & Collision type\\
  PERSONCOUNT & Total number of people involved in the collision\\
  PEDCOUNT & Number of pedestrians involved in the collision\\
  PEDCYLCOUNT & Number of bicycles involved in the collision\\
  VEHCOUNT & Number of vehicles involved in the collision\\
  INJURIES & Number of total injuries in the collision\\
  SERIOUSINJURIES & Number of serious injuries in the collision\\
  FATALITIES & Number of fatalities in the collision\\
  INCDATE & Date of the incident\\
  INCDTTM & Time of the incident\\
  JUNCTIONTYPE & Category of junction at which collision took place\\
  SDOT\_COLCODE & Code given to the collision by SDOT\\
  SDOT\_COLDESC & Description of the collision by SDOT\\
  INATTENTIONIND & Whether or not collision was due to inattention\\
  UNDERINFL & Whether it was under the influence of drugs or alcohol\\
  WEATHER & Description of the weather conditions\\
  ROADCOND & Condition of the road during the collision\\
  LIGHTCOND & Light conditions during the collision\\
  PEDROWNOTGRNT & Whether the pedestrian right of way was not granted\\
  SDOTCOLNUM & Number given to the collision by SDOT\\
  SPEEDING & Whether speeding was a factor in the collision\\
  ST\_COLCODE & Code provided by the state that describes the collision\\
  ST\_COLDESC & Description corresponding to the state’s coding scheme\\
  SEGLANEKEY & Key for the lane segment in which the collision occurred\\
  CROSSWALKKEY & Key for the crosswalk at which the collision occurred\\
  HITPARKEDCAR & Whether the collision involved hitting a parked car\\
  \hline
  \end{tabular}
  \caption{Fields in the Dataset}\label{fields}
\end{table}

As the dataset has possibly been sourced from a database table, several unique identifiers and spatial features are present in the database which may be irrelevant in further statistical analysis. These fields are OBJECTID, INCKEY, COLDETKEY, INTKEY, SEGLANEKEY, CROSSWALKKEY, and REPORTNO. Other fields such as EXCEPTRSNCODE, SDOT\_COLCODE, SDOTCOLNUM and LOCATION and their corresponding descriptions (if any) are categorical but have a large number of distinct values that shall not be that much useful for analysis. The INCDATE and INCDTTM denote the date and the time of the incident but may not be of use in further analyses. The data needs to be pre-processed.

\subsection{Statistical Insights}

\section{Methodology}

\section{Results}

\section{Conclusion}

\section{Future Work}

\end{document} 